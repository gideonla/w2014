\documentclass{article}
\usepackage{geometry}
\usepackage[namelimits,sumlimits]{amsmath}
\usepackage{amssymb,amsfonts}
\usepackage{multicol}
\usepackage{graphicx}
\usepackage{mathtools}
\usepackage[cm]{fullpage}
\newcommand{\tab}{\hspace*{5em}}
\newcommand{\conj}{\overline}
\newcommand{\dd}{\partial}
\newcommand{\ep}{\epsilon}
\newcommand{\openm}{\begin{pmatrix}}
\newcommand{\closem}{\end{pmatrix}}
\DeclareMathOperator{\cov}{cov}
\DeclareMathOperator{\rank}{rank}
\DeclareMathOperator{\im}{im}
\DeclareMathOperator{\lock}{lock}
\DeclareMathOperator{\unlock}{unlock}
\DeclareMathOperator{\Span}{span}
\DeclareMathOperator{\Null}{null}
\newcommand{\nc}{\newcommand}
\newcommand{\rn}{\mathbb{R}}
\newcommand{\zn}{\mathbb{Z}}
\nc{\cn}{\mathbb{C}}
\nc{\ssn}[1]{\subsubsection*{#1}}
\nc{\inner}[2]{\langle #1,#2\rangle}
\nc{\h}[1]{\widehat{#1}}
\nc{\tl}[1]{\widetilde{#1}}
\nc{\norm}[1]{\left\|{#1}\right\|}
\DeclarePairedDelimiter\ceil{\lceil}{\rceil}
\DeclarePairedDelimiter\floor{\lfloor}{\rfloor}
\begin{document}
Name: Hall Liu

Date: \today 
\vspace{1.5cm}

\subsection*{2}
\begin{enumerate}
    \item Liveness. The ordering ensures that no customer will be left hanging in the queue.
    \item Safety. Ensures that nothing goes into orbit accidentally.
    \item Liveness. Some thread will get to execute its critical section.
    \item Liveness. Messages will be printed promptly.
    \item Safety. Messages will not be lost.
    \item Safety. Cost of living will be monotonically increasing.
    \item Safety. There will never not be death or taxes.
    \item Safety. Harvard men will never be confused for something else?
\end{enumerate}
\subsection*{3}
Bob has a string tied to a can on Alice's side and vice versa. Start with an empty yard and Bob's side's can down. When Bob wants to put food out, he does the following:
\begin{enumerate}
    \item Wait until his can gets knocked down.
    \item Put food out.
    \item Go home and puts his can back up.
    \item Knock down Alice's can.
\end{enumerate}
When Alice wants to go get food, she does the following:
\begin{enumerate}
    \item Wait until her can gets knocked down.
    \item Go get food.
    \item Go home and put her can back up.
    \item Knock down Bob's can.
\end{enumerate}
Suppose that Bob's can is down. Then, the most recent action that can have happened to Alice's can is that it was put back up, since Bob is the only one who can knock down Alice's can and he will always reset his can before he knocks hers down. Similarly, if Alice's can is down, then Bob's must be up. Thus, both cans cannot be down simultaneously. Now, since both of them will only go outside when their can is down, they cannot be outside simultaneously. This satisifies mutual exclusion.

Suppose that Bob is always willing to provide food and that Alice is always trying to get food. If there is starvation, then either Bob or Alice must be waiting for their can to go down indefinitely. However, if only one of them is waiting, then the other one will complete their task and knock the first person's can down, so for starvation to occur, both cans must be up simultaneously and both parties must be waiting for them to fall. However, this state cannot be reached from the initial state. Since every time either party puts their can up they also knock down the other party's can, the total number of cans up must remain at $1$.

Finally, we have that Alice will never enter the yard unless there is food. This is because she will only enter the yard when her can is down, and her can can only be down when Bob has placed food in the yard and knocked it down after the last time she had retrieved food and reset her can. This is therefore a solution to the producers-consumers problem.
\subsection*{4}
\ssn{a}
Find some prisoner and assign him as the counter. Each time he goes into the room and sees the light on, he turns it off and adds 1 to his internal counter which he starts at 0. For every other prisoner, they will turn the light on the first time they see the light off, then always do nothing afterwards. When the counter reaches $P-1$, he will declare that all prisoners have visited. To show that this is a winning strategy, we need to show that the declaration is always correct and that the strategy will terminate. 

If the counter is at $P-1$, that means that the light has been turned on $P-1$ times in the past, since the light gets turned off every time the counter is incremented. Since each prisoner will only turn on the light once, this means that $P-1$ prisoners have visited, and the counter is guaranteed to have visited because he's been there at least $P-1$ times. To show that it will terminate, suppose that the counter $C$ is bounded above by some $Y<P-1$ for all time. Then, at some point, we will have $C=Y$ and the light off, and the counter in the room. At this point, only $Y$ prisoners have seen the light off when they came in, since otherwise the counter would be higher. Thus, since every prisoner will visit again after this point, at some point a prisoner will flip the light on, after which point the counter will come in again at some point and increment, contradicting the bound.
\ssn{b}
Pick a counter as before. When the counter enters the room, he will change the state of the light, and if the light is different from when he last left it, he will increment his counter. When any other prisoner enters, they will leave the light alone unless it is off and they have seen it in both positions, in which case they will flip the light on then rest forever. The counter declares the game over when the counter reaches $P-1$.

We want to show that whenever the counter gets incremented, it becomes equal to the number of prisoners who are now resting forever. Since the counter is always toggling the light on and off, the light won't get stuck in a steady state, and every prisoner will eventually get to see the light in both states and therefore get the chance to rest forever. Thus, if this holds, the counter will eventually get incremented to $P-1$. 

Let $C$ denote the newly incremented value of the counter and $R$ denote the number of prisoners resting forever at that point. Proceed by induction. Suppose that on the last visit of the counter, the light was left on. Then, no intervening prisoners will have touched the light, and the counter will flip the light off on this visit and not increment. Thus, the previous visit by the counter must have left the light in an off state. Now, if an intervening prisoner had turned the light on, he is the first and only prisoner in this par of the sequence to turn the light on and move into a resting-forever state, so the counter is rightly incremented by one.
\subsection*{5}
The prisoner in the back of the line should say ``red'' if the number of prisoners he can see with red hats is odd, and ``blue'' otherwise. Then, the next prisoner, knowing the parity of the number of red hats and the colors of everyone except himself, can deduce his own color. Similarly, each successive prisoner will know the colors of the people behind him (assume that they answered correctly by induction) and the colors of the people ahead of him, so everyone will be able to deduce their own color.
\subsection*{7}
We have $S_2=\frac{1}{1-p+p/2}$, or $p=2+\frac{2}{S_2}$. Plugging this into $S_n$ gives $\frac{1}{1-(2+2/S_2)+\frac{2+2/S_2}{n}}=\frac{S_2}{2\frac{S_2+1}{n}-(S_2+2)}$
\subsection*{8}
The speed of the 10-core processor compared to that of one of its cores will be $\frac{1}{1-p+\frac{p}{10}}$. Since the uniprocessor is $5$ times faster than a single core, the 10-core machine will be preferable iff $\frac{1}{1-p+\frac{p}{10}}>5$, or when $p>\frac{8}{9}$.
\subsection*{11}
This satistfies mutual exclusion. First, we show that we cannot transition from a state where no threads are in a critical section to one where more than one thread is. To show this, suppose that thread $n$ is the first to call \verb|lock|, then thread $m$ is the next to call \verb|lock| at some point after thread $n$ does. We have that 
\begin{verbatim}
write(n, turn=n) -> read(n, busy == false) -> write(n, busy=true) -> read(n, turn == n) -> CS(n)
write(m, turn=m) -> read(m, busy == false) -> write(m, busy=true) -> read(m, turn == m) -> CS(m)
\end{verbatim}
We must also have 
\begin{verbatim}
read(m, busy == false) -> write(n, busy=true)| and \verb|read(n, busy == false) -> write(m, busy=true)
\end{verbatim}
since nothing can flip \verb|busy| back to false (as nothing was in the critical section beforehand, so \verb|unlock| won't be called). Thus, transitivity gives
\begin{verbatim}
write(n, turn=n) -> ... -> read(m, turn == m)
write(m, turn=m) -> ... -> read(n, turn == n)
\end{verbatim}
which is impossible without intervening writes to \verb|turn| by either thread $n$ or $m$.

This does not satisfy deadlock-freedom. To see this, suppose thread $A$ is executing the inner do-loop and thread $B$ has just exited the inner do-loop. Thread $B$ sets \verb|busy| to true, then thread $A$ sets \verb|turn| to its own id. Then, thread $B$ will go back into the inner do-loop and never exit, and the same with thread $A$, because \verb|busy| will then be stuck in the on state. 
\subsection*{14}
Remove the last $l$ levels from the filter hierarchy and place the critical section directly underneath the $(n-l)$th level. Then, lemma 2.4.1 guarantees $l$-mutual-exclusion, and the proof of 2.4.2 will still work with the $(n-l)$th level as the base case.
\subsection*{15}
This fails to provide mutual exclusion. Suppose we have the following sequence of events:
\begin{verbatim}
write(1, x=1) -> write(2, x=2) -> read(1, y == -1) -> 
read(2, y == -1) -> write(1, y == 1) -> write(2, y == 2)
\end{verbatim}
At this point, thread 2 will return from the fast lock function because it reads that $x$ is 2, and thread $1$ will go ahead and call the slow lock function (which will return because it is uncontested), so both threads will enter the critical section.
\subsection*{16}
In order to get \verb|STOP|, a thread $k$ must follow this sequence:
\begin{verbatim}
write(k, last=k) -> read(k, goRight == false) -> write(k, goRight == true) -> read(k, last == k)
\end{verbatim}
Note that nothing will set \verb|goRight| to false after it has been set to true, so if some other thread $j$ is to follow this sequence, it must have \verb|read(j, goRight == false) -> write(k, goRight == true) -> read(k, last == k)|, and by transitivity, \verb|write(j, last=j) -> ... -> read(k, last == k)|, which is impossible. Thus only one thread will get \verb|STOP|.

For \verb|RIGHT|, since \verb|goRight| is initialized as false, the first thread to reach line 11 will read \verb|goRight == false|, which means that it cannot execute the statement inside the if, which happens to be the only place where a \verb|RIGHT| can be returned. Thus no more than $n-1$ threads can get \verb|RIGHT|.

For \verb|DOWN|, consider what happens to the thread that does not get \verb|RIGHT|. For notation's sake, label it thread 1. The sequence that it must follow to get \verb|DOWN| is 
\begin{verbatim}
write(1, last == 1) -> read(1, goRight == false) -> write(1, goRight=true) -> read(1, last != 1) -> DOWN
\end{verbatim}
This means that some other thread (call it 2) must have written to \verb|last| after thread $1$ did, or 
\begin{verbatim}
write(1, last=1) -> write(2, last=2) -> read(1, last != 1)
\end{verbatim}
If thread 2 also gets \verb|DOWN|, then there must be some thread 3 such that 
\begin{verbatim}
write(2, last=2) -> write(3, last=3) -> read(2, last != 2)
\end{verbatim}
Continuing in this manner for all $n$ threads to get \verb|DOWN|, we must then have 
\begin{verbatim}
write(1, last=1) -> ... -> write(n-1, last=n-1) -> write(n, last=n)
\end{verbatim}
However, for thread $n$ to get \verb|DOWN|, it must read \verb|last| to be something other than $n$, which is impossible because all write operations by the other threads to \verb|last| have already been completed. Thus not all threads can get \verb|DOWN|.
\end{document}
