\documentclass{article}
\usepackage{geometry}
\usepackage[namelimits,sumlimits]{amsmath}
\usepackage{amssymb,amsfonts}
\usepackage{multicol}
\usepackage{algpseudocode}
\usepackage{graphicx}
\usepackage[cm]{fullpage}
\newcommand{\tab}{\hspace*{5em}}
\newcommand{\conj}{\overline}
\newcommand{\dd}{\partial}
\newcommand{\ep}{\epsilon}
\newcommand{\openm}{\begin{pmatrix}}
\newcommand{\closem}{\end{pmatrix}}
\DeclareMathOperator{\cov}{cov}
\DeclareMathOperator{\rank}{rank}
\DeclareMathOperator{\im}{im}
\DeclareMathOperator{\Span}{span}
\DeclareMathOperator{\Null}{null}
\newcommand{\nc}{\newcommand}
\newcommand{\rn}{\mathbb{R}}
\nc{\cn}{\mathbb{C}}
\nc{\ssn}[1]{\subsubsection*{#1}}
\nc{\inner}[2]{\langle #1,#2\rangle}
\nc{\h}[1]{\widehat{#1}}
\nc{\tl}[1]{\widetilde{#1}}
\nc{\norm}[1]{\left\|{#1}\right\|}
\nc{\lb}{\lambda}
\nc{\ddx}[1]{\frac{d}{d{#1}}}
\nc{\ddxt}[1]{\frac{d^2}{d{#1}^2}}
\begin{document}

Name: Hall Liu

Date: \today 
\vspace{1.5cm}

\subsection*{1}
\ssn{a}
The first derivative of $\phi_2$ with respect to $\lb$ is $\frac{\ddx{\lb}\|p(\lb)\|}{\|p(\lb)\|^2}$. Differentiating again gives a $\|p(\lb)\|^2$ in the denominator (which we don't care about because it doesn't change sign), and the numerator is
\[ \|p(\lb)\|^2\ddxt{\lb}\|p(\lb)\|-\left(\ddx{\lb}\|p(\lb)\|\right)^2\cdot2\|p(\lb)\|=\|p(\lb)\|\left(\|p(\lb)\|\ddxt{\lb}\|p(\lb)\|-2\left(\ddx{\lb}\|p(\lb)\|\right)^2\right)\]
From (4.39) in the text, we have $\|p(\lb)\|=\sqrt{\sum\frac{(q_i^Tg)^2}{(\lb_i+\lb)^2}}$. Since $(q_i^Tg)^2$ is constant wrt $\lb$, denote it by $b_i$. Then, we have
\[\ddx{\lb}\|p(\lb)\|=-\frac{\sum\frac{b_i}{(\lb_i+\lb)^3}}{\|p(\lb)\|}\]
Taking the second derivative, we have
\[-\frac{1}{\|p(\lb)\|}\cdot-3\sum\frac{b_i}{(\lb_i+\lb)^4}+\frac{1}{\|p(\lb)\|^2}\left(\sum\frac{b_i}{(\lb_i+\lb)^3}\right)\frac{\sum\frac{b_i}{(\lb_i+\lb)^3}}{\|p(\lb)\|}\]
Simplifying, this becomes
\[\frac{3}{\|p(\lb)\|}\sum\frac{b_i}{(\lb_i+\lb)^4}+\frac{1}{\|p(\lb)\|}\left(\ddx{\lb}\|p(\lb)\|\right)^2\]
so the expression inside the parens in the numerator of the second derivative of $\phi_2$ is
\[3\sum\frac{b_i}{(\lb_i+\lb)^4}-\frac{1}{\|p(\lb)\|^2}\left(\sum\frac{b_i}{(\lb_i+\lb)^3}\right)^2\]
Thus, the second derivative will be nonnegative everywhere if we have that 
\[\left(\sum\frac{b_i}{(\lb_i+\lb)^3}\right)^2-\left(\sum\frac{b_i}{(\lb_i+\lb)^4}\right)\left(\sum\frac{b_i}{(\lb_i+\lb)^2}\right)\leq0\]
Expanding out the products and subtracting, the remaining terms are of the form 
\[\frac{2b_ib_j}{(\lb_i+\lb)^3(\lb_j+\lb)^3}-\frac{b_ib_j}{(\lb_i+\lb)^4(\lb_j+\lb)^2}-\frac{b_ib_j}{(\lb_i+\lb)^4(\lb_j+\lb)^2}\]
Factoring out the $b_ib_j$ (they're all positive), we're left with an expression of the form $2a^3b^3-a^4b^2-a^2b^4$, where $a,b\geq0$. In turn, this can be written as $-(a-b)^2a^2b^2\leq0$, which means that the whole thing is less than $0$, which means that the second derivative is nonnegative.
\ssn{b}
%Expressing the ``tangent below the graph'' property in the terms of this problem, we have that $\phi_2(\lb_l)-\phi_2'(\lb_l)(\lb_{l+1}-\lb_l)\leq\phi_2(\lb_{l+1})$. Substituting in $\lb_{l+1}=\lb_l-\frac{\phi_2(\lb_l)}{\phi_2'(\lb_l)}$, we have
%\[\phi_2(\lb_l)-\phi_2'(\lb_l)\left(-\frac{\phi_2(\lb_l)}{\phi_2'(\lb_l)}\right)\leq\phi_2(\lb_{l+1})\]
We know that $\phi_2$ is nonincreasing, so $\phi_2'(\lb)\leq0$ everywhere. If $\lb_{l}<\lb^*$, then $\phi_2(\lb_l)>0$, so $\lb_{l+1}=\lb_l-\frac{\phi_2(\lb_l)}{\phi_2'(\lb_l)}>\lb_l$. Then, using the ``tangent-below-the-graph'' property $\phi_2(\lb_l)+\phi_2'(\lb_l)(\lb^*-\lb_l)\leq\phi_2(\lb^*)=0$, we have that $\phi_2'(\lb_l)\leq-\frac{\phi_2(\lb_l)}{\lb^*-\lb_l}$, so that $\lb_l-\frac{\phi_2(\lb_l)}{\phi_2'(\lb_l)}\leq\lb_l+\frac{\phi_2(\lb_l)}{\frac{\phi_2(\lb_l)}{\lb^*-\lb_l}}=\lb^*$.

Otherwise if $\lb_l>\lb^*$, then the inequality $\phi_2(\lb_l)+\phi_2'(\lb_l)(\lb^*-\lb_l)\leq0$ turns into $\phi_2'(\lb_l)\geq-\frac{\phi_2(\lb_l)}{\lb^*-\lb_l}$. Since we now have $\phi_2(\lb_l)<0$, then $\lb_l-\frac{\phi_2(\lb_l)}{\phi_2'(\lb_l)}\leq\lb_l+\frac{\phi_2(\lb_l)}{\frac{\phi_2(\lb_l)}{\lb^*-\lb_l}}=\lb^*$
\ssn{c}
First, note that the ``otherwise'' condition will be hit only finitely times -- if the initial point is to the right of $\lb^*$, then that condition will be executed until $\tl{\lb}^{l+1}$ ends up to the right of $-\lb_1$ or until the iteration of the second condition takes $\lb^l$ to the left of $\lb^*$ (which is guaranteed to occur in a finite number of steps). Then, once the point is in $(-\lb_1,lb^*)$, all subsequent points will lie in that interval also by (b). Thus, all we need to worry about in the tail of the sequence is the Newton's method updating.

We already know that the algorithm converges to something in $(-\lb_1, \lb^*]$ due to the sequence being monotonic and bounded. Suppose that the sequence converges to something less than $\lb^*$, call it $x^*$. Let $\phi_2(x^*)=M>0$, and let $N$ be a uniform bound on $|\phi_2'(\lb)|$ on some open interval $I$ containing $x^*$. Then, we have for any $\lb^l\in I$, $\lb^l<x^*$ that $\lb^{l+1}=\lb^l+\frac{\phi_2(\lb^l)}{|\phi_2(\lb^l)|}\geq\lb^l+\frac{M}{N}$. Then, for $\lb^l$ sufficiently close to $x^*$, $\lb^{l+1}>x^*$, contradicting convergence.

To show quadratic convergence, for any $l$, we have by Taylor's theorem that
\[0=\phi_2(\lb^*)=\phi_2(\lb^l+(\lb^*-\lb^l))=\phi_2(\lb^l)+\phi_2'(\lb^l)(\lb^*-\lb^l)+\frac{1}{2}(\lb^*-\lb^l)^2\phi_2''(t)\]
for some $t\in[\lb^l,\lb^*]$. Then, dividing by $\phi_2'(\lb^l)(\lb^*-\lb^l)^2$ and rearranging gives
\[0=\frac{\frac{\phi_2(\lb^l)}{\phi_2'(\lb^l)}+\lb^*-\lb^l}{(\lb^*-\lb^l)^2}+\frac{\phi_2''(t)}{\phi_2'(\lb^l)}\]
The first term is the ratio that we care about -- we want to show that it's bounded above. Fortunately, the second derivative is uniformly bounded above on $(\lb^k, lb^*)$ where $\lb^k$ is the first iterate in $(-\lb_1,lb^*)$, and the magnitude of first derivative is uniformly bounded below on the same interval. Thus, we have quadratic convergence.
\ssn{d}
To solve the subproblem, we are given a matrix $B$ and the value of the gradient and the function at the particular point. The characterization theorem gives us necessary and sufficient criteria for a solution. First, we want to check to see whether there is a solution with $\|p\|<\Delta$. If this is to be true, then we must have $\lb=0$, which means that $B$ must be pos. semidef. and there is a solution to $Bp=-g$. Since we're going to be using the eigenvalue decomposition, we can achieve this step as follows:

\begin{algorithmic}
    \State let $QDQ^T=B$
    \If{all entries of $D$ are nonnegative}
        \State attempt to compute a solution to $DQ^Tp=-Q^Tg$
        \If{the zeros in $D$ are in the same place as in $Q^Tg$}
            \If{the solution has norm less than or equal to $\Delta$}
                \State return solution
            \EndIf
        \EndIf
    \EndIf
    \State assert that $\|p\|=\Delta$ and proceed to iterative solution
\end{algorithmic}

We also need to ensure that $q_1^Tg\neq0$. If this doesn't hold, then we can just use (4.45) to compute $p$ and be done with it. Now, for the iteration, we have

\begin{algorithmic}
    \State Partition $Q$ by columns into $Q_1$ and $Q_2$, with $Q_1$ containing all eigenvectors with the lowest eigenvalue.
    \State Let $D$ be the array of eigenvalues in increasing order. Partition it similarly to $Q$.
    \If{$Q_1^Tg=0$}
        \State $u\gets Q_2^Tg$
        \State $d\gets D_2+\lambda_1$
        \State $\tau\gets\sqrt{\Delta-\sum\left(\frac{u_i}{d_i}\right)^2}$
        \State return $\sum_{i\in Q_2}\left(\frac{u_i}{d_i}\right)q_i+\tau Q_1(1)$
    \EndIf

    \State Initialize $\lb=\lb^0$, $c$.
    \While{$|\phi_2(\lb)|>\ep$}
        \State Compute $\phi_2(\lb)$ and $\phi_2'(\lb)$ according to formulas from (a)
        \State $\tl{\lb}\gets\lb-\frac{\phi_2(\lb)}{\phi_2'(\lb)}$
        \If{$\tl{\lb}\leq-\lb_1$}
            \State $\lb\gets-c\lb_1+(1-c)\lb$
        \Else
            \State $\lb\gets\tl{\lb}$
        \EndIf
    \EndWhile
    \State return $-\sum_{i=1}^n\frac{q_i^Tg}{\lambda_i+\lambda}q_j$
\end{algorithmic}
\ssn{e}
For the $6^4$ grid, the average time starting from $\openm1&2&3&4\closem^T$ were $1.27$s and $3.36\times10^{-4}$s for the function evaluation and the subproblem, resp. Starting from $\openm2&4&8&16\closem^T$ gives $1.27$s and $1.74$s. With the $3^4$ grid from $\openm1&2&3&4\closem^T$, the function evaluations took $7.19\times10^{-3}$s and the subproblem took $2.24\times10^{-4}$s. Starting from the other point, the times were $7.48\times10^{-3}$ and $1.81\times10^{-4}$ seconds, respectively. This is with the C version of the function evaluation code. With the pure Matlab implementation, the function evalution took about 30 times longer. Thus, our strategy for solving the subproblem really doesn't matter in this case, since minimizing the number of function evaluations is the priority for reducing runtime.
\subsection*{2}
The backtracking approach gives us a pair of inequalities. The first is identical to the Goldstein conditions:  $f(x_k+\alpha p_k)\leq f(x_k)+c\alpha\nabla f_k^Tp_k$. The second comes from the termination of the iteration: $f(x_k+\alpha p_k/\rho)>f(x_k)+c\alpha\nabla f_k^Tp_k/\rho$, where $\rho\in(0,1)$. 

Define $g(t)=f(x_k+t\alpha p_k/\rho)$. Then $g(0)=f(x_k)$ and $g(1)=f(x_k+\alpha p_k/\rho)$. We have 
\[g(1)-g(0)=\int_0^1\frac{d}{dt}g(t)dt=\int_0^1\alpha\nabla f(x_k+t\alpha p_k/\rho)^Tp_k=\frac{\alpha}{\rho}\nabla f(x_k+t\alpha p_k/\rho)^Tp_k\]
for some $t\in[0,1]$, the latter equality by the mean value theorem. Plugging this into the second inequality from backtracking, we have
\[\alpha\nabla f(x_k+t\alpha p_k/\rho)^Tp_k>c\alpha\nabla f(x_k)^Tp_k\]
Subtracting $\alpha\nabla f(x_k)^Tp_k$ from both sides and collecting terms, we have
\[\alpha_k\left(\nabla f(x_k+t\alpha p_k/\rho)-\nabla f(x_k)\right)^Tp_k>\alpha(c-1)\nabla f(x_k)^Tp_k\]
Since the RHS is positive, we can apply Cauchy-Schwarz then the Lipschitz condition to the LHS to get
\[\frac{L\alpha^2t}{\rho}\|p_k\|^2>\alpha(c-1)\nabla f(x_k)^Tp_k\]
Since $t\in[0,1]$, we can discard it. Rearranging so that we get a bound on $\alpha$, we have 
\[\alpha>\frac{\rho(c-1)\nabla f(x_k)^Tp_k}{L\|p_k\|^2}\]
Then, plugging this into the first inequality gives us
\[f(x_k)-f(x_{k+1})\geq-c\alpha\nabla f(x_k)^Tp_k>-\frac{\rho(c^2-c)\left(\nabla f(x_k)^Tp_k\right)^2}{L\|p_k\|^2}\]
Now since $a^Tb=\|a\|\|b\|\cos(\theta)$, we have that the above is equal to 
\[\frac{\rho(c-c^2)\|\nabla f(x_k)\|^2\cos^2(\theta_k)}{L}\]
Thus, the sum of the above expression over all $k$ is bounded above by the sum of $f(x_k)-f(x_{k+1})$ over all $k$(ignoring the constant terms $\rho$ and $c$), which is a bounded telescoping sum due to the below-boundedness of $f$. Thus, we have convergence.
\subsection*{3}
We know from the structure of the trust region and line search algorithms that the sequence of function values is monotonic -- $f(x_{k+1})\leq f(x_k)$ for all $k$. The second part of the assumption gives us that there is a unique, isolated minimizer $x^*$ somewhere in $L$. Then, we know that if $f(x_k)$ converges to $f(x^*)$, we have by continuity of $f$ that $x_k$ converges to $x^*$.

Translate the function such that $f(x^*)=0$. Suppose that $f(x_k)$ does not converge to $0$. Then, since the sequence of function values is monotonic, we must have that $f(x_k)$ is bounded below by some $\ep>0$, which means that all $x_k$ remain outside the set $\{x: f(x)\leq\ep/2\}$. However, since there is a limit point $x^\sharp$ of the $x_k$ with $\|\nabla f(x^\sharp)\|=0$, this means that $f(x^\sharp)\geq\ep/2$ and $\|\nabla f(x^\sharp)\|=0$, which contradicts the strong convexity that the assumption gives us.
\subsection*{4}
\ssn{a}
Differentiating $P(x)$ wrt $d$, we have $f'(x)+df''(x)+\frac{1}{2}d^2f'''(x)$. Finding the roots of this (the stationary points) with the quadratic formula, we have 
\[d=\frac{-f''(x)+\sqrt{\left(f''(x)\right)^2-2f'(x)f'''(x)}}{f'''(x)}\]
As $x\to x^*$, the term inside the square root approaches $\left(f''(x)\right)^2$ because $f'''(x)$ and $f''(x)$ are both bounded both above and away from zero near $x^*$. Thus, we have a stationary point $d$ that approaches zero as $x\to x^*$
\ssn{b}
I'm going to reformulate this problem as finding a point $x$ at which $g(x)\equiv f'(x)=0$ so that I can save a few tickmarks. Then, the iteration equation becomes $x_{n+1}=x_n+\frac{-g'(x_n)+\sqrt{\left(g'(x_n)\right)^2-2g(x_n)g''(x_n)}}{g''(x_n)}$. 

Let $d(x_n)$ be defined as the RHS of the above. Then, we have that $d(x^*)=x^*$ and $x_{n+1}=d(x_n)$. Taylor-expand $d(x_n)$ from $x^*$ to get
\[d(x_n)=d(x^*)+d'(x^*)(x_n-x^*)+\frac{d''(x^*)}{2}(x_n-x^*)^2+\frac{d'''(\alpha)}{6}(x_n-x^*)^3\]
If we define $e_n=x_n-x^*$, then after moving $d(x^*)=x^*$ to the LHS, we get
\[e_{n+1}=d'(x^*)e_n+\frac{d''(x^*)}{2}e_n^2+\frac{d'''(\alpha)}{6}e_n^3\ (*)\]
Now, let's calculate some derivatives. The first derivative of $d$ with respect to $x$ is
\[1+\frac{g''(x)\left(-g''(x)+\frac{2g'(x)g''(x)-2g'(x)g''(x)-2g(x)g'''(x)}{2\sqrt{g'(x)^2-2g(x)g''(x)}}\right)-(-g'(x)+\sqrt{g'(x)^2-2g(x)g''(x)})g'''(x)}{g''(x)^2}\]
Plugging in $x^*$ to this expression, the $g(x)$ terms all become zero, and hence we have
\[1+\frac{-g''(x^*)^2+0-(-g'(x)+g'(x))g'''(x)}{g''(x^*)^2}=0\]
so the first term on the RHS of $(*)$ disappears.

Simplifying the first derivative so that we can differentiate again, we get
\[-\frac{g(x)g'''(x)}{g''(x)\sqrt{g'(x)^2-2g(x)g''(x)}}+\frac{g'(x)g'''(x)}{g''(x)^2}-\frac{g'''(x)\sqrt{g'(x)^2-2g(x)g''(x)}}{g''(x)^2}\]
Differentiate this (and plugging in zeros where appropriate) to get
\[-\frac{(g''(x^*)\sqrt{g'(x^*)^2-0})(g'(x^*)g'''(x^*)+0)-0}{g''(x^*)^2g'(x^*)^2-0}+\frac{g''(x^*)^3g'''(x^*)+g''(x^*)^2g'(x^*)g''''(x^*)-2g'(x^*)g'''(x^*)g''(x^*)g'''(x^*)}{g''(x^*)^4}\]
\[-\frac{g''(x^*)^2\left(g''''(x^*)\sqrt{g'(x^*)^2-0}+0\right)+2g''(x^*)g'''(x^*)^2\sqrt{g'(x^*)^2-0}}{g''(x^*)^4}\]
which simplifies down to 
\[-\frac{g'''(x^*)g''(x^*)^2}{g''(x^*)^3}+\frac{g''(x^*)^2g'''(x^*)}{g''(x^*)^3}+\frac{g'(x^*)g''''(x^*)}{g''(x^*)^2}+\frac{2g'(x^*)g'''(x^*)^2}{g''(x^*)^3}-\frac{g''''(x^*)g'(x^*)}{g''(x^*)^2}+\frac{2g'(x^*)g'''(x^*)^2}{g''(x^*)^3}\]
Yay, it all cancels out to zero. That means the second term in $(*)$ goes away too, and we're left with
\[e_{n+1}=\frac{d'''(\alpha)}{6}e_n^3\]
Dividing and taking absolute values on both sides, we get cubic convergence (assuming that $d'''$ is bounded in a neighborhood near $x^*$)
\ssn{c}
The hypothesis for larger order approximations would be that an order $n$ Taylor approximation yields order $n$ convergence for the optimization problem. Of course, for $n>5$, we run into problems because of Abel-Ruffini, which means that such a method would be wildly impractical. For higher dimensions, the same order of convergence is expected. The value of $d$ would be a solution to the minimization of $f(x)+\nabla f(x)^Td+\frac{1}{2}d^T\nabla^2 f(x)d+\frac{1}{6}\sum_i b_i b^TT_ib$, where $T_i$ is the $i$th slice of the three-dimensional grid of the third partial derivatives of $f$.
\end{document}

