\documentclass{article}
\usepackage{geometry}
\usepackage[namelimits,sumlimits]{amsmath}
\usepackage{amssymb,amsfonts}
\usepackage{multicol}
\usepackage{graphicx}
\usepackage[cm]{fullpage}
\newcommand{\tab}{\hspace*{5em}}
\newcommand{\conj}{\overline}
\newcommand{\dd}{\partial}
\newcommand{\ep}{\epsilon}
\newcommand{\openm}{\begin{pmatrix}}
\newcommand{\closem}{\end{pmatrix}}
\DeclareMathOperator{\cov}{cov}
\DeclareMathOperator{\rank}{rank}
\DeclareMathOperator{\im}{im}
\DeclareMathOperator{\Span}{span}
\DeclareMathOperator{\Null}{null}
\newcommand{\nc}{\newcommand}
\newcommand{\rn}{\mathbb{R}}
\nc{\cn}{\mathbb{C}}
\nc{\ssn}[1]{\subsubsection*{#1}}
\nc{\inner}[2]{\langle #1,#2\rangle}
\nc{\h}[1]{\widehat{#1}}
\nc{\tl}[1]{\widetilde{#1}}
\nc{\norm}[1]{\left\|{#1}\right\|}
\begin{document}

Name: Hall Liu

Date: \today 
\vspace{1.5cm}

\subsection*{3}
Let $g(t)=f(x_k+t\alpha_kp_k)$. Then, $g(0)=f(x_k)$ and $g(1)=f(x_{k+1})$, and we have 
\[g(1)-g(0)=\int_0^1\frac{d}{dt}g(t)dt=\int_0^1\alpha_k\nabla f(x_k+t\alpha_kp_k)^Tp_k\]
By the mean value theorem, there exists some $t\in[0,1]$ such that the above is equal to $\alpha_k\nabla f(x_k+t\alpha_kp_k)^Tp_k$. Then, by the left side of the Goldstein inequality, we have
\[\alpha_k\nabla f(x_k+t\alpha_kp_k)^Tp_k\geq(1-c)\alpha_k\nabla f(x_k)^Tp_k\implies c\alpha_k\nabla f(x_k)^Tp_k\geq\alpha_k(\nabla f(x_k)^Tp_k-\nabla f(x_k+t\alpha_kp_k)^Tp_k)\]
Since $\nabla f(x_k)^Tp_k<0$, we flip the signs around and get by the Lipschitz condition:
\[-c\alpha_k\nabla f(x_k)^Tp_k\leq Lt\alpha_k^2\|p_k\|^2\implies \alpha_k\geq-\frac{c\nabla f(x_k)^Tp_k}{Lt\|p_k\|^2}\geq-\frac{c\nabla f(x_k)^Tp_k}{L\|p_k\|^2}\]
the last inequality being because $0<t<1$.

Now, plugging this into $f(x_{k+1})-f(x_k)\leq c\alpha_k\nabla f(x_k)^Tp_k$, we have
\[f(x_{k})-f(x_{k+1})\geq-c\alpha_k\nabla f(x_k)^Tp_k\geq\frac{(c\nabla f(x_k)^Tp_k)^2}{L\|p_k\|^2}\]
Using the identity $a^Tb=\|a\|\|b\|\cos(\theta)$, we have that the above is equal to 
\[\frac{c^2\|\nabla f(x_k)^T\|^2\cos^2(\theta_k)}{L}\]
Thus, the sum of the above expression over all $k$ is bounded above by the sum of $f(x_k)-f(x_{k+1})$ over all $k$, which is a bounded telescoping sum due to the below-boundedness of $f$. Thus, we have convergence.
\end{document}
